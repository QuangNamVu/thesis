\chapter{Giới thiệu} 
\section{Giới thiệu đề tài nghiên cứu}
Hiện nay, tiền mã hóa đã trở nên phổ biến, đa dạng với nhiều sàn giao dịch khác nhau. Đồng tiền mã hóa có tỷ giá thay đổi theo thời gian, việc tìm xu hướng giá ngắn hạn tính theo giờ phút không bị ảnh hưởng nhiều bởi các yếu tố bên ngoài như các dự báo, các quy định của chính phủ. Từ dữ liệu cụ thể là tổng hợp của các giao dịch trên các sàn trực tuyến việc tìm ra một giải thuật có thể dự đoán xu hướng giá của các giao dịch tiếp theo với nguyên tắc đề cao khách quan so với kinh nghiệm bản thân là một vấn đề mới mẻ. Vậy nên tôi quyết định chọn đề tài \textbf{Dự đoán xu hướng giá ngắn hạn các đồng tiền mật mã bằng kĩ thuật học máy}.
\section{Mục tiêu và phạm vi đề tài}
\subsection{Mục tiêu}

Mục tiêu của luận văn này là xây dựng một công cụ dự đoán xu hướng giá ngắn hạn các đồng tiền mật mã bằng kĩ thuật học máy. Dữ liệu đầu vào là các thông tin về lịch sử giá các  đồng tiền ảo trong các phiên giao dịch.

\subsection{Phạm vi đề tài}
\begin{itemize}
\item Tìm hiểu và nghiên cứu về lý thuyết học máy thống kê (statistical machine learning)
\item Xây dựng mô hình dự đoán vế xu hướng tăng giảm, dự đoán giá của các đồng trong thời gian ngắn hạn.
\end{itemize}
Các đối tượng nghiên cứu trong đề tài:
\begin{itemize}
\item Tìm hiểu một vài loại đồng tiền mã hóa, và các sàn giao dịch.
\item Một vài tài liệu liên quan tới lý thuyết thống kê hiện đại.
\item Tìm hiểu một vài mô hình trong học máy: hồi quy logistic, rừng ngẫu nhiên, mạng nơron.
\item Sử dụng ngôn ngữ Python, R và một số thư viện để hiện thực mô hình.
\item Xây dựng công cụ dự đoán giá một cách tự động.
\end{itemize}

\section{Tiến độ thực hiện}
Trong phần này, tác giả xin trình bày lịch trình công việc đã thực hiện đề tài trong học
kỳ I và lịch trình dự kiến hiện thực đề tài trong quá trình làm luận văn chính thức ở học
kỳ II dưói dạng biểu đồ Gantt sau đây.

\input{figures/gantt_plan}