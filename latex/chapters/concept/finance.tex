\section{Các khái niệm về tài chính}
\subsection{Tính thanh khoản (Liquidity)}
% https://www.binance.vision/economics/liquidity-explained

% Liquidity as a term is defined as the ability to buy or sell assets in the market without causing a drastic change in the assets price.
Khái niệm về tính thanh khoản dùng để chỉ mức độ mà một tài sản có thể được mua hoặc bán trên thị trường mà không làm ảnh hưởng nhiều đến giá thị trường.
% Liquidity can refer to two different areas; liquid market and liquid asset.
% Liquid market means that there are always investors in the market willing to trade. A liquid asset refers to an asset that can be easily converted into cash.
Khái niệm tính thanh khoản được chia thành 2 loại: tính thanh khoản thị trường (liquid market) và tính thanh khoản về tài sản (liquid asset). Thị trường có tính thanh khoản cao ám chỉ rằng trong thị trường thường xuyên có các nhà đầu tư sẵn sàng giao dịch. Một tài sản có tính thanh khoản cao đồng nghĩa với việc tài sản đó có thể chuyển đổi sang tiền mặt một cách dễ dàng. Đối với thị trường tiền mã hóa, để so sánh tính thanh khoản giữa các sàn trong cùng một thời điểm hoặc tính thanh khoản của một sàn tại những thời điểm khác nhau có 3 yếu tố quan trọng: 
%24 hour trading volume, Order book depth and the amount by which the ask price exceeds the bid price, also known as the bid/ask spread.
\begin{itemize}
	\item Lượng đồng giao dịch trong ngày.
	\item Số lượng lệnh mua/bán dựa trên danh sách lệnh (order book) được công khai dựa theo các sàn như Coinbase Pro \cite{live-order-book}, Binance, Bittrex, \dots
	\item Lượng chênh lệch giữa giá yêu cầu của bên bán và giá bỏ thầu của bên mua (bid/ask spread).
\end{itemize}

\subsection{Nhiễu (Noise)} \label{concept:finance:noise}
Khái niệm nhiễu có quan hệ đối lập với khái niệm thông tin (information) với dữ liệu giá cung cấp đầy đủ thông tin, việc dự đoán dễ dàng và ngược lại với dữ liệu có nhiễu cao do bị ảnh hưởng bởi các yếu tố khác như phần đề cập tại phần \ref{overview:factor}. Nhiễu khiến những quan sát của các nhà đầu tư không được hoàn hảo, điều này dẫn thị trường có khả năng lưu động\cite{noise-in-finance}.
