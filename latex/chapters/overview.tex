\chapter{Tổng quan về lĩnh vực nghiên cứu} 
\section{Những yếu tố tác động đến giá trị đồng tiền mã hóa} \label{overview:factor}
\subsection{Cung và cầu của thị trường}
Trong nguyên tắc chính của kinh tế nếu người ta mua một đồng tiền, giá trị của đồng tiền sẽ tăng lên và nếu người ta bán đồng tiền, giá sẽ giảm.
\subsection{Tin tức trên các phương tiện thông tin đại chúng}
Các sự kiện chính trị và kinh tế trên toàn thế giới ảnh hưởng đến cách mà con người phản ứng với các dự đoán giá, tin tức cảnh báo về rủi ro tác động chính lên cung-cầu.
\subsection{Quy định của chính phủ}
Có 4 cấp độ quản lý tiền ảo hiện nay đang được các nước thực thi, cụ thể:
\begin{itemize}
\item Cấm trên diện rộng.
\item Cấm trong lĩnh vực tài chính ngân hàng (trong đó có Trung Quốc, Nga).
\item Cảnh báo rủi ro đối với người sử dụng, đầu tư.
\item  Chấp nhận như một phương tiện thanh toán (các nước chấp nhận đồng bitcoin gồm có Mỹ, Canada, Úc, Liên minh châu Âu, Phần Lan \cite{CountriesAllowBTC} ).
\end{itemize}
\subsection{Chính sách của các tổ chức}
Facebook, Google và Twitter đã ngăn chặn khách hàng và người dùng sử dụng dịch vụ cryptocurrency.
\subsection{Các vấn đề kỹ thuật}
Vì đồng tiền mã hóa có thể bị hack thành công vào tài khoản hoặc tấn công máy chủ, có thể làm giảm tỷ giá hối đoái, dẫn đến giá giảm.
\section{Nhu cầu sử dụng tiền mã hoá của mỗi hệ sinh thái}
\begin{itemize}
    \item Số thành viên tham gia vào hệ sinh thái (Số người đến khu vui chơi mua vé tham gia các trò chơi trong đó bằng tiền A).
    \item Số lượng dịch vụ trong hệ sinh thái (Khu vui chơi có càng nhiều trò chơi thì nhu cầu sử dụng tiền A càng tăng); Và các nền tảng như Ethereum luôn mở cho các đối tác tạo các dịch vụ gia tăng trên đó giống như khu vui chơi cho phép đối tác bên ngoài vào tổ chức trò chơi ở trong.
    \item  Số người đầu cơ: Những người nhận thấy nhu cầu tiền mã hoá của một hệ sinh thái tăng dần sẽ mua để nắm giữ chờ tăng giá thì bán ra. (Giống như phe vé bóng đá ngày trước mua vé chờ sát trận nhu cầu tăng vọt thì bán ra. Khu vui chơi thì ít có nhóm này vì lượng vé không bị giới hạn).
    \item  Số người bán bên ngoài chấp nhận tiền mã hoá: Một số người bán nhận thấy tính thanh khoản của tiền mã hoá và giá trị tăng dần của nó nên đã chấp nhận khách hàng thanh toán các hàng hoá dịch vụ của mình bằng loại tiền này (Nhà hàng bên cạnh khu vui chơi có thể chấp nhận khách hàng thanh toán bằng tiền A).
\end{itemize}