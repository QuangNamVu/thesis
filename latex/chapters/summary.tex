\chapter{Thí nghiệm và đánh giá} \label{chap-Implement}
Trong hai chương trước có đề cập tới các mô hình tham khảo và cách thu thập, xử lý dữ liệu để chuẩn bị cho quá trình thí nghiệm. Tiếp sau đây, trong chương này sẽ trình bày thư viện sử dụng, mô tả tham số trong mô hình. Dựa trên kết quả các mô hình từ đó rút ra các nhận xét về mô hình sử dụng và các điểm hạn chế khi hiện thực đồng thời đưa ra các hướng phát triển của luận văn.

\section{Các độ đo được sử dụng}
Trong phần này, chúng tôi trình bày các độ đo được sử dụng để đánh giá các mô hình. Trước
hết, chúng tôi trình bày ma trận nhầm lẫn (confusion matrix) cho các nhãn dự đoán như trong Bảng 
% \ref{tab:confusion_matrix} 
\narrowlinespacing

\begin{table}[H] 
    \centering

    \begin{tabularx}{0.9\textwidth}{
    p{\dimexpr.2\linewidth-2\tabcolsep-1.0\arrayrulewidth}% column 1
    p{\dimexpr.2\linewidth-2\tabcolsep-1.0\arrayrulewidth}% column 2
    p{\dimexpr.25\linewidth-2\tabcolsep-1.0\arrayrulewidth}
    p{\dimexpr.25\linewidth-2\tabcolsep-1.0\arrayrulewidth}
    }
        
        \toprule\midrule
        \textbf{} & \textbf{} & \multicolumn{2}{c}{\textbf{Kết quả dự đoán}} \\
        % \midrule
        \cmidrule(rl){3-4}
        \textbf{} & \textbf{} & \textbf{Giá tăng} & \textbf{Giá giảm} \\
        \midrule
        \multirow{2}{*}{\textbf{Nhãn thực tế }}  & \textbf{Giá tăng} & True Positive (TP) & 
        False Positive (FP)\\
        & \textbf{Giá giảm} & False Negative (FN) & True Negative (TN) \\
        % \hline
        \midrule
        \bottomrule
        
    \end{tabularx}
    \label{tab:confusion_matrix}
    \caption{Ma trận nhầm lẫn  cho các nhãn dữ liệu}
\end{table}

\normallinespacing


% \section{Thí nghiệm Rừng ngẫu nhiên}
% Trong quá trình huấn luyện, mô hình rừng ngẫu nhiên cho kết quả gần như tuyệt đối, tuy nhiên khi kiểm định trên các tháng cuối mô hình đạt khoảng 53-55\% với nhiều lần chạy. Với kết quả trên ngoài các độ đo như f1-score, độ chính xác, đề tài sử dụng cách kiểm định cross-validation nhằm xác định độ sai khác mẫu (pattern) giữa các phiên giao dịch. Kết quả 
%  \narrowlinespacing

\begin{table}[H] 
    \centering

    \begin{tabularx}{0.7\textwidth}{
    p{\dimexpr.2\linewidth-2\tabcolsep-1.0\arrayrulewidth}% column 1
    p{\dimexpr.2\linewidth-2\tabcolsep-1.0\arrayrulewidth}% column 2
    p{\dimexpr.15\linewidth-2\tabcolsep-1.0\arrayrulewidth}
    p{\dimexpr.15\linewidth-2\tabcolsep-1.0\arrayrulewidth}
    }
        
        \toprule\midrule
        \textbf{} & \textbf{} & \multicolumn{2}{c}{\textbf{Kết quả dự đoán}} \\
        % \midrule
        \cmidrule(rl){3-4}
        \textbf{} & \textbf{} & \textbf{Giá giảm} & \textbf{Giá tăng} \\
        \midrule
        % \hline
        
        %[[682 423]
        %   [393 662]] -> 0.6222222222222222
        
%         % 3 months
        % [[7592 5462]
        %  [4420 8446]] -> 0.61875
        \multirow{2}{*}{\textbf{Nhãn thực tế }}  & \textbf{Giá giảm} & 487 & 562\\
        & \textbf{Giá tăng} & 388 & 723 \\
        
        % [[487 562]
        %  [388 723]]
        % \hline
        \midrule
        \bottomrule
        
    \end{tabularx}
    \label{tab:rf_confusion_matrix}
    \caption{Ma trận nhầm lẫn từ mô hình rừng ngẫu nhiên}
\end{table}

\normallinespacing

%  Kết quả cho độ chính xác là 56.02\%. \\
 
% \section{Thí nghiệm mô hình Máy học véctơ hỗ trợ}

% \textbf{Mô hình SVM} Kết quả được tính bằng trung bình của 3 tháng cuối tương ứng với 2160 phiên liên tục. Kết quả được mô tả theo bảng dưới đây:\\

% % \narrowlinespacing

\begin{table}[H] 
    \centering

    \begin{tabularx}{0.7\textwidth}{
    p{\dimexpr.2\linewidth-2\tabcolsep-1.0\arrayrulewidth}% column 1
    p{\dimexpr.2\linewidth-2\tabcolsep-1.0\arrayrulewidth}% column 2
    p{\dimexpr.15\linewidth-2\tabcolsep-1.0\arrayrulewidth}
    p{\dimexpr.15\linewidth-2\tabcolsep-1.0\arrayrulewidth}
    }
        
        \toprule\midrule
        \textbf{} & \textbf{} & \multicolumn{2}{c}{\textbf{Kết quả dự đoán}} \\
        % \midrule
        \cmidrule(rl){3-4}
        \textbf{} & \textbf{} & \textbf{Giá giảm} & \textbf{Giá tăng} \\
        \midrule
        % \hline
        
        % [[555 494]
        %  [473 638]]
        
%         % 3 months
%         [[6700 6354]
%           [3840 9026]]
        \multirow{2}{*}{\textbf{Nhãn thực tế }}  & \textbf{Giá giảm} & 555 & 494\\
        & \textbf{Giá tăng} & 473 & 638 \\
        % \hline
        \midrule
        \bottomrule
        
    \end{tabularx}
    \label{tab:svm_confusion_matrix}
    \caption{Ma trận nhầm lẫn từ mô hình SVM}
\end{table}

\normallinespacing


% \narrowlinespacing

\begin{table}[H] 
    \centering

    \begin{tabularx}{0.7\textwidth}{
    p{\dimexpr.2\linewidth-2\tabcolsep-1.0\arrayrulewidth}% column 1
    p{\dimexpr.2\linewidth-2\tabcolsep-1.0\arrayrulewidth}% column 2
    p{\dimexpr.15\linewidth-2\tabcolsep-1.0\arrayrulewidth}
    p{\dimexpr.15\linewidth-2\tabcolsep-1.0\arrayrulewidth}
    }
        
        \toprule\midrule
        \textbf{} & \textbf{} & \multicolumn{2}{c}{\textbf{Kết quả dự đoán}} \\
        % \midrule
        \cmidrule(rl){3-4}
        \textbf{} & \textbf{} & \textbf{Giá giảm} & \textbf{Giá tăng} \\
        \midrule
        % \hline
        
        % [[555 494]
        %  [473 638]]
        
%         % 3 months
%         [[6700 6354]
%           [3840 9026]]
        \multirow{2}{*}{\textbf{Nhãn thực tế }}  & \textbf{Giá giảm} & 555 & 494\\
        & \textbf{Giá tăng} & 473 & 638 \\
        % \hline
        \midrule
        \bottomrule
        
    \end{tabularx}
    \label{tab:svm_confusion_matrix}
    \caption{Ma trận nhầm lẫn từ mô hình SVM}
\end{table}

\normallinespacing

    
% Kết quả trên đưa ra độ chính xác 55.23\%

% \section{Thí nghiệm mô hình Hồi quy Logistic}

%  \narrowlinespacing

\begin{table}[H] 
    \centering

    \begin{tabularx}{0.7\textwidth}{
    p{\dimexpr.2\linewidth-2\tabcolsep-1.0\arrayrulewidth}% column 1
    p{\dimexpr.2\linewidth-2\tabcolsep-1.0\arrayrulewidth}% column 2
    p{\dimexpr.15\linewidth-2\tabcolsep-1.0\arrayrulewidth}
    p{\dimexpr.15\linewidth-2\tabcolsep-1.0\arrayrulewidth}
    }
        
        \toprule\midrule
        \textbf{} & \textbf{} & \multicolumn{2}{c}{\textbf{Kết quả dự đoán}} \\
        % \midrule
        \cmidrule(rl){3-4}
        \textbf{} & \textbf{} & \textbf{Giá giảm} & \textbf{Giá tăng} \\
        \midrule
        % \hline
        
        % [[605 444]
        %  [508 603]] -> 0.5592592592592592
        
%         % 3 months
        % [[8069 4985]
        %  [4284 8582]] -> 0.6423996913580247
        \multirow{2}{*}{\textbf{Nhãn thực tế }}  & \textbf{Giá giảm} & 605 & 444\\
        & \textbf{Giá tăng} & 508 & 603 \\
        % \hline
        \midrule
        \bottomrule
        
    \end{tabularx}
    \label{tab:lr_confusion_matrix}
    \caption{Ma trận nhầm lẫn từ mô hình hồi quy logistic}
\end{table}

\normallinespacing

 
%  \textbf{Mô hình hồi quy logistic} Kết quả cho độ chính xác là 55.93\%.
 
% \section{Thí nghiệm mô hình dựa trên VAE}

% \narrowlinespacing

\begin{table}[H] 
    \centering
    \caption{Ma trận nhầm lẫn từ mô hình VAE}
    \begin{tabularx}{0.7\textwidth}{
    p{\dimexpr.20\linewidth-2\tabcolsep-1.0\arrayrulewidth}% column 1
    p{\dimexpr.20\linewidth-2\tabcolsep-1.0\arrayrulewidth}% column 2
    p{\dimexpr.15\linewidth-2\tabcolsep-1.0\arrayrulewidth}
    p{\dimexpr.15\linewidth-2\tabcolsep-1.0\arrayrulewidth}
    }
        
        \toprule\midrule
        \textbf{} & \textbf{} & \multicolumn{2}{c}{\textbf{Kết quả dự đoán}} \\
        % \midrule
        \cmidrule(rl){3-4}
        \textbf{} & \textbf{} & \textbf{Giá giảm} & \textbf{Giá tăng} \\
        \midrule
        % \hline
        
% [[566 481]
%  [496 617]] -> 0.5476851851851852


%         % 3 months
%         [[6700 6354]
%           [3840 9026]]
        \multirow{2}{*}{\textbf{Nhãn thực tế }}  & \textbf{Giá giảm} & 566 & 481\\
        & \textbf{Giá tăng} & 496 & 617 \\
        % \hline
        \midrule
        \bottomrule
        
    \end{tabularx}
    \label{tab:vae_confusion_matrix}
\end{table}

 
%  \textbf{Mô hình dựa trên VAE} Kết quả cho độ chính xác là 54.76\%.
%  \subsection{Áp dụng dữ liệu vào mô hình mạng nơron tích chập}
%  Đề tài có sử dụng thư viện hỗ trợ keras trên nền tensorflow một thư viện mã nguồn mở có hỗ trợ khả năng tính toán của các bộ xử lý đồ họa. Trong bước tiền xử lý dữ liệu để đưa vào trong mạng có sử dụng mỗi cửa sổ trượt làm một ảnh một chiều với số kênh là số thuộc tính của mỗi giao dịch, độ dài của mỗi ảnh được định nghĩa trước là số giao dịch liên tục. Nhãn của những ảnh này là xu hướng tăng hoặc giảm của giao dịch cuối cùng với mỗi ảnh.\\
%  Khi dữ liệu được đưa vào mạng nơron cần được chuẩn hóa để tránh hiện tượng các nơron không cập nhật được khi hàm kích hoạt có họ ReLU hoặc khó kích hoạt khi các hàm kích hoạt phi tuyến khác như hàm sigmoid hoặc hàm tanh.\\ Các bước \textbf{chuẩn hóa dữ liệu} được mô tả như sau:
%  \begin{itemize}
%  	\item Các thuộc tính về thời gian đổi về dạng số nguyên theo chuẩn UNIX.Các số này khá lớn nên được chuẩn hóa dạng logarit, sau đó chuẩn hóa theo standard score.
%  	%TODO Standard score viết công thức
%  	\item Các thuộc tính còn lại như lượng giao dịch, giá mở, giá đóng theo dạng standard score
%  \end{itemize}

% Dữ liệu được đưa vào mạng nơron tích chập với ý tuởng chính như sau:

\section{Kết quả}
\subsection{So sánh kết quả các mô hình đề xuất}

Chúng tôi so sánh kết quả dự đoán của bốn mô hình được sử dụng trong luận văn trên tập dữ liệu giao dịch từ ngày 2017/08/17 gồm hai khung thời gian phiên là 1 giờ và 5 phút cho ra các kết quả trong bảng:
\begin{table}[H] 
    \centering

    \begin{tabularx}{0.8\textwidth}
    {
    p{\dimexpr.4\linewidth-2\tabcolsep-1.0\arrayrulewidth}% column 1
    p{\dimexpr.2\linewidth-2\tabcolsep-1.0\arrayrulewidth}% column 2
    p{\dimexpr.2\linewidth-2\tabcolsep-1.0\arrayrulewidth}
    }
        
        \toprule
        \textbf{Mô hình} & \textbf{Độ chính xác} & \textbf{f1 score} \\
        % \cmidrule
        \midrule
        \textbf{Rừng ngẫu nhiên} & \textbf{rf} & \textbf{rf} \\
        \textbf{SVM} & \textbf{svm} & \textbf{svm} \\
        \textbf{Hồi quy Logistic} & \textbf{lr} & \textbf{lr} \\
        \textbf{VAE} & \textbf{vae} & \textbf{vae} \\
        \bottomrule
        
    \end{tabularx}
    \label{tab:1h_compare}
    \caption{Kết quả dự đoán trên 1 giờ (đơn vị: \%)}
\end{table}



\begin{table}[H] 
    \centering

    \begin{tabularx}{0.8\textwidth}
    {
    p{\dimexpr.4\linewidth-2\tabcolsep-1.0\arrayrulewidth}% column 1
    p{\dimexpr.2\linewidth-2\tabcolsep-1.0\arrayrulewidth}% column 2
    p{\dimexpr.2\linewidth-2\tabcolsep-1.0\arrayrulewidth}
    }
        
        \toprule
        \textbf{Mô hình} & \textbf{Độ chính xác} & \textbf{f1 score} \\
        % \cmidrule
        \midrule
        \textbf{Rừng ngẫu nhiên} & 61.88 & \textbf{rf} \\
        \textbf{SVM} & 60.67 & svm \\
        \textbf{Hồi quy Logistic} & 64.24 & \textbf{lr} \\
        \textbf{VAE} & \textbf{vae} & \textbf{vae} \\
        \bottomrule
        
    \end{tabularx}
    \label{tab:1h_compare}
    \caption{Kết quả dự đoán trên 5 phút (đơn vị: \%)}
\end{table}
% TODO: nhận xét




% : Rừng ngẫu nhiên, Máy học véctơ hỗ trợ, hồi quy logistic và mô hình dựa trên VAE




% \begin{algorithm}[H]
	\KwData{507918 giao dịch bitcoin/Yên}
	\KwResult{Mô hình với các tham số, độ chính xác khi test}
	\textbf{Khởi tạo:}\\
	\begin{itemize}
		\item Chuẩn hóa dữ liệu theo từng cột.
		\item Tập train: 480000 bộ cửa sổ đầu, mỗi cửa sổ chứa 100 giao dịch liên tục nhau, mỗi cửa sổ liên tiếp nhau 1 giao dịch.
		\item Các tham số của mô hình tại mỗi lớp.
		\item $\ell=\infty$
	\end{itemize}
	\While{Khi số lượng bộ test nhỏ hơn 1024}{
		Tập test : sau 100 giao dịch tiếp theo so với tập train lấy 1024 bộ cửa sổ liên tiếp.\;
		Lấy tập test làm tập kiểm định;
		Tính $\ell$ là Loss của mô hình đối với tập kiểm định;

		\If{$\ell_1 < \ell$}{
			Cập nhật tham số\; $\ell = \ell_1$\;
		}
		Tập train: tăng kích thước của tập train thêm lên 1024 cửa sổ tiếp theo.
	}
	\caption{Áp dụng kĩ thuật rolling window}
\end{algorithm}

% Khi áp dụng standard score, với mỗi thuộc tính có giá trị trung bình về 0 và phương sai về 1. Điều này ảnh hưởng tốt cho mạng nơron tích chập khi hội tụ nhanh và khó bị `kẹt lại' ở những điểm thung lũng hơn.

% % \dots
%  % TODO dụng dữ liệu thô như hình 1.1
%   \subsection{Áp dụng dữ liệu vào mô hình mạng Long Short-term Memory}
%  Mô hình mạng LSTM có ưu điểm 
%  Các số liệu kết quả trong phần này được lấy từ các lần thực nghiệm huấn luyện trên google colab, một nền tảng dịch vụ có hỗ trợ phần cứng và thư viện miễn phí, với dữ liệu đầu vào từ  xxxxxxxxx và độ chính xác theo bài toán phân loại lên hoặc xuống.
 
 

 

  
  %\includegraphics{figures/} 
 % TODO: get information about google clould resource