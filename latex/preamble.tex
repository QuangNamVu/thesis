\usepackage{babel}
\usepackage[utf8]{inputenc}
\usepackage[T1]{fontenc}
\usepackage{vntex}
\renewcommand{\baselinestretch}{1.2} 
\PassOptionsToPackage{table}{xcolor}
\usepackage[table]{xcolor}
%\usepackage[utf8]{inputenc}
%\usepackage[francais]{babel}
\usepackage{geometry}
\usepackage{a4wide,amssymb,epsfig,latexsym,multicol,array,hhline}
\usepackage{lastpage}
\usepackage{emptypage}
\usepackage{pdfpages}
\usepackage[lined,boxed,commentsnumbered]{algorithm2e}
\usepackage{enumerate}
\usepackage{color}
\usepackage{spverbatim}
\usepackage{graphicx}							% Standard graphics package
\usepackage{array}
\usepackage{tabularx}
\usepackage{multirow}
\usepackage{multicol}
\usepackage{rotating}
\usepackage{graphics}
\usepackage[labelsep=endash]{caption}

% keep figure in subsection
\usepackage[section]{placeins} 
\usepackage{flafter}

\usepackage{setspace}
\usepackage{epsfig}
\usepackage{tikz}
\usetikzlibrary{arrows,snakes,backgrounds}
\usepackage{hyperref}
% check reference if not cite
%\usepackage{refcheck}
\usepackage{cleveref}
\hypersetup{urlcolor=blue,linkcolor=black,citecolor=black,colorlinks=true} 
\usepackage{shapepar}
\usepackage{titlesec}
\usepackage{mdframed}
\usepackage{mathtools}
\usepackage{amsmath}
\usepackage{amssymb}
\usepackage{changepage}
% pseudocode
\usepackage[lined,boxed,commentsnumbered]{algorithm2e}
\usepackage{tabularx, caption}
\usepackage{qtree}
\usepackage{pgfgantt}
\usepackage{pgfplots}
\usepackage{tkz-euclide}
\usepackage{accents}
\usepackage{setspace}
\usepackage{animate}
\usepackage{lipsum}
\usepackage{verbatim}

\usetikzlibrary{arrows,snakes,backgrounds,automata,positioning,trees,shapes,calc,through,bending}
\usetikzlibrary{datavisualization}
\usetikzlibrary{datavisualization.formats.functions}
\usepackage{movie15}
\usepackage{pst-plot}
\usepackage[numbered]{bookmark}
\usepackage{tocloft}
\usepackage{helvet}
\usepackage{listings}
\usepackage[ddmmyyyy]{datetime}
\usepackage{csvsimple}
\usepackage{float}
\usepackage{cases}
%\usepackage[nomessages]{fp}
\usetkzobj{all} 
%\usepackage{pstcol} 								% PSTricks with the standard color package

%\usepackage[backend=bibtex,style=authoryear,natbib=true]{biblatex}
%\usepackage[autostyle=true]{csquotes}
%,citestyle=numeric
\usepackage[backend=bibtex]{biblatex} % Use the bibtex backend with the authoryear citation style (which resembles APA)

\addbibresource{bibliography.bib}

%\usepackage[autostyle=true]{csquotes} % Required to generate language-dependent quotes in the bibliography

\newtheorem{theorem}{{\bf Định lý}}
\newtheorem{constraint}{{\bf Ràng buộc}}
\newtheorem{property}{{\bf Tính chất}}
\newtheorem{proposition}{{\bf Mệnh đề}}
\newtheorem{corollary}[proposition]{{\bf Hệ quả}}
\newtheorem{lemma}[proposition]{{\bf Bổ đề}}


%%ensembles de nombres
\def\NP{$\mathcal{NP}$}
\def\N{\mathbb{N}}
\def\Z{\mathbb{Z}}
\def\R{\mathbb{R}}
\def\Q{\mathbb{Q}}
\addto\captionsenglish{% Replace "english" with the language you use
  \renewcommand{\contentsname}{Mục lục}
  \renewcommand{\listfigurename}{Danh mục hình ảnh}
  \renewcommand{\chaptername}{Chương}
  \renewcommand{\figurename}{Hình}
  \renewcommand{\abstractname}{Tóm tắt}
  \renewcommand{\bibname}{Tài liệu tham khảo}  
  \renewcommand{\refname}{Tài liệu tham khảo}  
  %
}
\newcolumntype{M}[1]{>{\centering\arraybackslash}m{#1}}
\newcolumntype{P}[1]{>{\centering\arraybackslash}p{#1}}
%\def\myd#1{\text{\d{\ensuremath#1}}}


\newcounter{numproblem}
\newenvironment{problem}{\addtocounter{numproblem}{1}
\noindent{\large \bf Problem \thenumproblem. }}{}

\newcounter{numexercise}
\newenvironment{exercise}{\addtocounter{numexercise}{1}
\noindent{\large \bf Exercise \thenumexercise. \\}}{}

\newcounter{numquestion}
\newenvironment{question}{\addtocounter{numquestion}{1}
\noindent{\large \bf Question \thenumquestion. \\}}{}


\newcounter{numcau}
\newenvironment{cau}{\addtocounter{numcau}{1}
\noindent{\large \bf Câu \thenumcau. \\}}{}

%\newcounter{numsolution}
\newenvironment{solution}{
\noindent{ \large \bf Solution.\\}  \color{blue}}{~~\hfill$\Box$\\}

\newenvironment{loigiai}{
\noindent{ \large \bf Lời giải.\\}  \color{blue}}{~~\hfill$\Box$\\}

\newif\ifshortversion
\shortversiontrue 	% hide this line when shortversion==false
\newcommand\version[2]{\ifshortversion #1 \else #2 \fi}

\newcommand{\tikzAngleOfLine}{\tikz@AngleOfLine}
\def\tikz@AngleOfLine(#1)(#2)#3{%
\pgfmathanglebetweenpoints{%
\pgfpointanchor{#1}{center}}{%
\pgfpointanchor{#2}{center}}
\pgfmathsetmacro{#3}{\pgfmathresult}%
} 


%\usepackage{fancyhdr}
\definecolor{darkblue}{RGB}{3, 43, 145}
\definecolor{lightblue}{RGB}{20, 136, 219}
\definecolor{amber}{rgb}{1.0, 0.75, 0.0}
\definecolor{aqua}{rgb}{0.0, 1.0, 1.0}
\definecolor{dkgreen}{rgb}{0,0.6,0}
\definecolor{gray}{rgb}{0.5,0.5,0.5}
\definecolor{mauve}{rgb}{0.58,0,0.82}
\definecolor{awesome}{rgb}{1.0, 0.13, 0.32}
\definecolor{banana}{rgb}{1.0, 0.88, 0.21}
\definecolor{brilliantlavender}{rgb}{0.96, 0.73, 1.0}
\definecolor{bubbles}{rgb}{0.91, 1.0, 1.0}
\newsavebox\LogoHCMUT
\savebox\LogoHCMUT{%
\centering
\renewcommand{\familydefault}{\sfdefault}
\normalfont
\begin{tikzpicture}[shorten >=1pt,node distance=1.5cm,on grid,auto,/tikz/initial text=,font=\small,align=center] 
 \newdimen\R
 \newdimen\fsbk
 \newdimen\fshcm
   \R=142pt
   \fsbk=135pt
   \fshcm=50pt
   \coordinate (O) at (0:0) {};
   \node (BK) at ($(0:0)+(90:\R/7)$)[draw=none] {\color{darkblue} \bf 
   \begingroup
    \fontsize{\fsbk}{\fsbk}\selectfont
     BK
	\endgroup
	\\};
   \node  (HCM) at ($(BK.south)+(-90:\R/7)$)[draw=none] {\color{lightblue} \bf 
   \begingroup
    \fontsize{\fshcm}{\fshcm}\selectfont
    \centering
    \hspace{-1pt} TP.HCM
	\endgroup
	\\};
   \foreach \x in {0,60,...,360}  {\draw[draw = none] ($(O)+(\x+30:\R)$) -- ($(O)+(\x+90:\R)$);}
   \foreach \i in {0,120,240} {
    \draw[fill=darkblue,draw = none] ($(O)+(\i+150:\R)$) -- ($(O)+(\i+150:\R)+(\i+90:\R)$) -- ($(O)+(\i+90:\R)+(\i+90:\R)$) -- ($(O)+(\i+90:\R)$) -- cycle;
   \draw[fill=lightblue,draw = none] ($(O)+(\i+30:\R)$) -- ($(O)+(\i+30:\R)+(\i+90:\R)$) -- ($(O)+(\i+90:\R)+(\i+90:\R)$) -- ($(O)+(\i+90:\R)$) -- cycle;
   }
\end{tikzpicture}
\renewcommand{\familydefault}{\rmdefault}
\normalfont
}
\newcommand{\insertlogo}[2]{\resizebox{#1}{#2}{\usebox{\LogoHCMUT}}}%



\newbox\one
\newbox\two
\long\def\loremlines#1{%
    \setbox\one=\vbox {%
      \lipsum
     }
   \setbox\two=\vsplit\one to #1\baselineskip
   \unvbox\two}
   
   \setcounter{secnumdepth}{4}
\setcounter{tocdepth}{3}
\makeatletter
\newcounter {subsubsubsection}[subsubsection]
\renewcommand\thesubsubsubsection{\thesubsubsection .\@alph\c@subsubsubsection}
\newcommand\subsubsubsection{\@startsection{subsubsubsection}{4}{\z@}%
                                     {-3.25ex\@plus -1ex \@minus -.2ex}%
                                     {1.5ex \@plus .2ex}%
                                     {\normalfont\normalsize\bfseries}}
\newcommand*\l@subsubsubsection{\@dottedtocline{3}{10.0em}{4.1em}}
\newcommand*{\subsubsubsectionmark}[1]{}

%%% mathematic
\newcommand\givenbase[1][]{\:#1\lvert\:}
\let\given\givenbase
\newcommand\sgiven{\givenbase[\delimsize]}
\DeclareMathOperator*{\argmin}{argmin}
\DeclareMathOperator*{\argmax}{argmax}
%\DeclarePairedDelimiterX\Basics[1](){\let\given\sgiven #1}
%\newcommand\Average{E\Basics}

%%%%%%%%%%%%%%%%%%%%%%%%%%%%%%%%%%%%%%%%%%%%
%% Dkl
%\DeclarePairedDelimiterX{\infdivx}[2]{(}{)}{%
%	#1\;\delimsize\|\;#2%
%}
%\newcommand{\Dkl}{D_{KL}\infdivx}
%\DeclarePairedDelimiter{\norm}{\lVert}{\rVert}



\makeatother


%%%%%%%%%%%%%%%%%%%%%%%%%%%%%%%%%%%%%%


%\newcommand{\tikzAngleOfLine}{\tikz@AngleOfLine}
\def\tikz@AngleOfLine(#1)(#2)#3{%
\pgfmathanglebetweenpoints{%
\pgfpointanchor{#1}{center}}{%
\pgfpointanchor{#2}{center}}
\pgfmathsetmacro{#3}{\pgfmathresult}%
} 

\def\roundloop[#1]#2#3{%
 \coordinate (rla) at (#2.east); 
 \path   (#2)--++(#1) coordinate (rlb);
 \tkzTgtFromP(#2,rla)(rlb)            
 \node (rlb) at (rlb) [circle through={(tkzFirstPointResult)}] {};
 \coordinate  (rlc) at (intersection 2 of #2 and rlb);
 \coordinate  (rld) at (intersection 1 of #2 and rlb);         
 \tikzAngleOfLine(rlb)(rld){\AngleStart}
 \tikzAngleOfLine(rlb)(rlc){\AngleEnd} 
 \tikzAngleOfLine(#2)(rlb){\AngleLabel}
 \ifdim\AngleStart pt<\AngleEnd pt
 \draw[thick,->]%
   let \p1 = ($ (rlb) - (rld) $), \n2 = {veclen(\x1,\y1)}
   in   
     (rlb) ++(\AngleLabel:\n2) node[]{#3}
     (rld) arc (\AngleStart:\AngleEnd:\n2); 
 \else 
  \draw[thick,->]%
   let \p1 = ($ (rlb) - (rld) $), \n2 = {veclen(\x1,\y1)}
   in   
     (rlb) ++(\AngleLabel:\n2) node[]{#3}
     (rld) arc (\AngleStart-360:\AngleEnd:\n2); 
   \fi 
  }
  
\graphicspath{{figures/}}

\newcommand{\executeiffilenewer}[3]{%
	\ifnum\pdfstrcmp{\pdffilemoddate{#1}}%
	{\pdffilemoddate{#2}}>0%
	{\immediate\write18{#3}}\fi%
}


\newcommand{\includesvg}[1]{%
	\executeiffilenewer{#1.svg}{#1.pdf}%
	{inkscape -z -D  --file=#1.svg --export-pdf=#1.pdf --export-latex}%
	\input{#1.pdf_tex}%
}
