\linespread{1.2}
\usepackage[vietnamese.licr, english]{babel}
\usepackage[vietnamese=nohyphenation]{hyphsubst}
\usepackage{url}
\usepackage{tikz}
\usetikzlibrary{calc}

\usepackage{multirow}
\usepackage{multicol}

%\usepackage[lined,boxed,commentsnumbered,algo2e]{algorithm2e}
%\usepackage[lined,boxed,commentsnumbered]{algorithm2e}
%\usepackage{algorithm,algpseudocode}



%\usepackage[{algorithm2e} 

\usepackage{tabularx}

\usepackage{amsmath}
\newcommand\numberthis{\addtocounter{equation}{1}\tag{\theequation}}

\usepackage{mathtools}
\usepackage{hyperref}

% numbering eq iff ref
%\mathtoolsset{showonlyrefs=true}
%\mathtoolsset{showonlyrefs}
%\mathtoolsset{showonlyrefs=true,showmanualtags=true}

\usepackage{graphicx,wrapfig,lipsum}
\usepackage[section]{placeins} 
%\graphicspath{{images/}}
\graphicspath{{./}}
\usepackage{parskip}
\usepackage{vmargin}
\usepackage{vntex}
\usepackage[utf8]{inputenc}
\usepackage[toc,page]{appendix}

\usepackage[lined,boxed,commentsnumbered]{algorithm2e}
\usepackage{algpseudocode}

%\usepackage{algpseudocode}

\usepackage{color, colortbl}

\usepackage{pgfgantt}
\usepackage{listings}
\usepackage{pgfplots}
\usepackage{subfig}
\usepackage{mwe}
\usepackage{chngcntr}
\usepackage{caption}
\usepackage{blindtext}
\usepackage{booktabs,array,ragged2e}
\newcolumntype{P}[1]{>{\RaggedRight\arraybackslash}p{#1}}
\newcommand{\tabitem}{\textbullet~~}

\usepackage{makecell}
\usepackage[flushleft]{threeparttable}

\usepackage{bookmark}

\usepackage[bitstream-charter]{mathdesign}
\usepackage[T1,T5]{fontenc}
\usepackage{titlesec}
\definecolor{myblue}{RGB}{0,82,155}

\titleformat{\chapter}[display]
  {\normalfont\bfseries\color{myblue}}
  {\filleft%
    \begin{tikzpicture}
    \node[
      outer sep=0pt,
      text width=2.5cm,
      minimum height=3cm,
      fill=myblue,
      font=\color{white}\fontsize{80}{90}\selectfont,
      align=center
      ] (num) {\thechapter};
    \node[
      rotate=90,
      anchor=south,
      font=\color{black}\Large\normalfont
      ] at ([xshift=-5pt]num.west) {\chaptertitlename};  
    \end{tikzpicture}%
  }
  {10pt}
  {\titlerule[2.5pt]\vskip3pt\titlerule\vskip4pt\LARGE}


\pgfplotsset{compat=1.8}
\counterwithin{figure}{chapter}
%\counterwithout{figure}{chapter}

\definecolor{dkgreen}{rgb}{0,0.6,0}
\definecolor{gray}{rgb}{0.5,0.5,0.5}
\definecolor{mauve}{rgb}{0.58,0,0.82}

%%%
\definecolor{barblue}{RGB}{133, 193, 233}
\definecolor{groupblue}{RGB}{27, 79, 114}
\definecolor{linkred}{RGB}{165,0,33}
%%%


%%% mathematic
\newcommand\givenbase[1][]{\:#1\lvert\:}
\let\given\givenbase
\newcommand\sgiven{\givenbase[\delimsize]}
\DeclareMathOperator*{\argmin}{argmin}
\DeclareMathOperator*{\argmax}{argmax}


%\DeclarePairedDelimiterX\Basics[1](){\let\given\sgiven #1}
%\newcommand\Average{E\Basics}

%%%%%%%%%%%%%%%%%%%%%%%%%%%%%%%%%%%%%%%%%%%
% Dkl
\DeclarePairedDelimiterX{\infdivx}[2]{(}{)}{%
	#1\;\delimsize\|\;#2%
}
\newcommand{\Dkl}{D_{KL}\infdivx}
\DeclarePairedDelimiter{\norm}{\lVert}{\rVert}


%%%%%%%%%%%%%%%%%%%%%%%%%%%%%%%%%%%%%%%%%%%

\makeatother


%%%%%%%%%%%%%%%%%%%%%%%%%%%%%%%%%%%%%%


%\newcommand{\tikzAngleOfLine}{\tikz@AngleOfLine}
\def\tikz@AngleOfLine(#1)(#2)#3{%
	\pgfmathanglebetweenpoints{%
		\pgfpointanchor{#1}{center}}{%
		\pgfpointanchor{#2}{center}}
	\pgfmathsetmacro{#3}{\pgfmathresult}%
} 

\def\roundloop[#1]#2#3{%
	\coordinate (rla) at (#2.east); 
	\path   (#2)--++(#1) coordinate (rlb);
	\tkzTgtFromP(#2,rla)(rlb)            
	\node (rlb) at (rlb) [circle through={(tkzFirstPointResult)}] {};
	\coordinate  (rlc) at (intersection 2 of #2 and rlb);
	\coordinate  (rld) at (intersection 1 of #2 and rlb);         
	\tikzAngleOfLine(rlb)(rld){\AngleStart}
	\tikzAngleOfLine(rlb)(rlc){\AngleEnd} 
	\tikzAngleOfLine(#2)(rlb){\AngleLabel}
	\ifdim\AngleStart pt<\AngleEnd pt
	\draw[thick,->]%
	let \p1 = ($ (rlb) - (rld) $), \n2 = {veclen(\x1,\y1)}
	in   
	(rlb) ++(\AngleLabel:\n2) node[]{#3}
	(rld) arc (\AngleStart:\AngleEnd:\n2); 
	\else 
	\draw[thick,->]%
	let \p1 = ($ (rlb) - (rld) $), \n2 = {veclen(\x1,\y1)}
	in   
	(rlb) ++(\AngleLabel:\n2) node[]{#3}
	(rld) arc (\AngleStart-360:\AngleEnd:\n2); 
	\fi 
}

%\graphicspath{{figures/}}
\graphicspath{{./}}

\newcommand{\executeiffilenewer}[3]{%
	\ifnum\pdfstrcmp{\pdffilemoddate{#1}}%
	{\pdffilemoddate{#2}}>0%
	{\immediate\write18{#3}}\fi%
}

\usepackage{svg}

% \newcommand{\includesvg}[1]{%
% 	\executeiffilenewer{#1.svg}{#1.pdf}%
% 	{inkscape -z -D  --file=#1.svg --export-pdf=#1.pdf --export-latex}%
% 	\input{#1.pdf_tex}%
% }


\newcommand{\normallinespacing}{\renewcommand{\baselinestretch}{1.5} \normalsize}
\newcommand{\mediumlinespacing}{\renewcommand{\baselinestretch}{1.2} \normalsize}
\newcommand{\narrowlinespacing}{\renewcommand{\baselinestretch}{1.0} \normalsize}


%% Json display

\colorlet{punct}{red!60!black}
\definecolor{background}{HTML}{EEEEEE}
\definecolor{delim}{RGB}{20,105,176}
\colorlet{numb}{magenta!60!black}

\lstdefinelanguage{json}{
	basicstyle=\normalfont\ttfamily,
	numbers=left,
	numberstyle=\scriptsize,
	stepnumber=1,
	numbersep=8pt,
	showstringspaces=false,
	breaklines=true,
	frame=lines,
	backgroundcolor=\color{background},
	literate=
	*{0}{{{\color{numb}0}}}{1}
	{1}{{{\color{numb}1}}}{1}
	{2}{{{\color{numb}2}}}{1}
	{3}{{{\color{numb}3}}}{1}
	{4}{{{\color{numb}4}}}{1}
	{5}{{{\color{numb}5}}}{1}
	{6}{{{\color{numb}6}}}{1}
	{7}{{{\color{numb}7}}}{1}
	{8}{{{\color{numb}8}}}{1}
	{9}{{{\color{numb}9}}}{1}
	{:}{{{\color{punct}{:}}}}{1}
	{,}{{{\color{punct}{,}}}}{1}
	{\{}{{{\color{delim}{\{}}}}{1}
	{\}}{{{\color{delim}{\}}}}}{1}
	{[}{{{\color{delim}{[}}}}{1}
	{]}{{{\color{delim}{]}}}}{1},
}


%% python highlight

\usepackage{pythonhighlight}